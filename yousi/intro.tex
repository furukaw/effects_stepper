\section{はじめに}
\label{sec:intro}

ステッパとはプログラミング教育やデバッグのために使うツールであり、プログラムが代数的に書き換わる様子を出力することで実行過程を見せるものである。

これまでに Racket \cite{JMM2001} や OCaml \cite{JT2019} のプログラミング教育用に制限した構文などを対象にステッパが作られてきたが、shift/reset \cite{OA1990} や algebraic effects and handlers \cite{何} (以下、algebraic effects) といった、継続を明示的に扱うための機能を含む言語のステッパは作られていない。継続を扱うプログラムの挙動を理解するのは困難なので、そういった言語に対応したステッパを作ることが本研究の目的である。

ステッパは簡約のたびにその時点でのプログラム全体を出力するインタプリタなので、実行している部分式のコンテキスト(周りの式)の情報が常に必要になる。継続を扱うような複雑な機能を持つ言語を対象にしたステッパでは、コンテキストがどのような構造をしているかが自明でない。そこで、通常のインタプリタ関数をプログラム変換することで機械的にコンテキストの情報を保持させ、ステッパを実装する方法を示す。実際に型無し$\lambda$計算 \cite{何} と algebraic effects から成る言語を対象にしてステッパを実装する過程を説明する。

