\section{他の言語への対応}
\label{section:languages}

\ref{section:definition} 節で示した algebraic effect handlers を含む言語の CPS インタプリタをステッパにするには、非関数化、CPS変換、非関数化が必要だったが、他のいくつかの言語についても同様にインタプリタを変換することでステッパを導出することを試みた。それぞれの言語のステッパ導出について説明する。


\subsection{型無し$\lambda$計算}
\label{subsection:lambda}

型無し$\lambda$計算の DS インタプリタは、CPS変換して非関数化したら全てのコンテキストを引数に保持するインタプリタになり、出力関数を入れるのみでステッパを作ることができた。これは、継続を区切って一部を捨てたり束縛したりするという操作が無いためである。


\subsection{try-with}
\label{subsection:try_with}

try-with は、algebraic effect handlers が限定継続を変数に束縛するのと違って、例外が起こされたときに限定継続を捨てるという機能である。よって継続を表す値は現れないので、インタプリタを CPS で書く必要は無い。Direct style でインタプリタを書いた場合、最初にCPS変換をすることで、CPSインタプリタと同様の変換によってステッパが導出できた。最初からCPSインタプリタを書いていれば \ref{section:transform}節と同様の手順になる。


\subsection{shift/reset}
\label{subsection:shift/reset}

shift/reset は algebraic effect handlers と同様に限定継続を変数に束縛して利用することができる機能である。\ref{section:transform} 節で行ったのと全く同様に、CPS インタプリタを非関数化、CPS変換、非関数化したらコンテキストが表れ、ステッパが得られた。


\subsection{Multicore OCaml}
\label{subsection:multicore_ocaml}

Multicore OCaml は、OCaml の構文に algebraic effect handlers を追加した構文を持つ。我々は \ref{section:transform} 節で得られたステッパをもとにして、Multicore OCaml の algebraic effect handlers を含む一部の構文を対象にしたステッパの実装を目指している。Multicore OCaml の「エフェクト」は \ref{section:definition} 節で定義した言語の algebraic effect handlers のオペレーションとほとんど同じものであり、継続が one-shot であることを除いて簡約のされかたは \ref{section:definition} 節で定めた言語のインタプリタと同様なので、インタプリタ関数を用意できれば変換によってステッパが導出できると考えられる。

このような構文の多い言語について実際にステッパを実装しようとすると、新たに問題が発生する。
我々が実装し大学の授業で利用している OCaml の一部の構文のステッパでは、例えば以下のように問題に対処した。
\begin{itemize}
\item プログラムが大きくなってしまい見づらくなる問題。
例えばリストの要素数が多い場合に \texttt{[1; 2; 3; ...]} というように、
冗長な式を省略表示することで見やすくした。
\item ステップ数が多くなり、見たいステップまで進むのが大変になる問題。
文の実行を始める際にマークを出力しておき、
そのマークがあるところの直後のステップに飛ばす機能を実装した。
1文の中でステップ数が多くなる場合については、
関数適用などをスキップする機能によって目的のステップまで早くたどり着けるようにしている。
\item ref (書き換え可能な変数) などをどう表示するべきかという問題。
書き換え可能な変数の値はそのアドレスを表す識別子とし、
その中身を取得する演算子を適用すると中身に簡約されるという扱いにした。
それぞれのステップを表す文字列に、その時点での状態(参照先の値やそれまでの標準出力内容)
を含めることで、各時点での値を確認でき、
また副作用が実行されるステップで状態が変化する様子を観察できるようにした。
例えば、プログラム \texttt{let a = ref 0 in !a} のステップ列は
\texttt{let a = ref 4 in !a $\leadsto$ (* \$1:\ 4 *) let a = \$1 in !a
$\leadsto$ \\(* \$1:\ 4 *) !\$1 $\leadsto$ (* \$1:\ 4 *) 4} となる。
\end{itemize}
これらについては Multicore OCaml のステッパでも同様の機能を作ることを考えている。
