\documentclass[twoside, twocolumn, a4paper]{jarticle}

%
% is.sty は必要
\usepackage{is}
%
%
%
% 以下各自が必要なパッケージは加えてよい.
%
%\usepackage{rsfs}
\usepackage{amsmath}
\usepackage{stmaryrd}
\usepackage{amsfonts}
\usepackage{amssymb}
\usepackage{theorem}
% \usepackage{epsf}

% 自分で定義したマクロファイル
% 型
\newcommand\Int{\textsf{int}}
\newcommand\Arrow[2]{{#1}\rightarrow{#2}}

% ラムダ式
\newcommand\Lam[2]{\lambda{#1}.\,{#2}}
\newcommand\LamP[2]{(\Lam{#1}{#2})}
\newcommand\App[2]{{#1}\,{#2}}
\newcommand\AppP[2]{(\App{#1}{#2})}
\newcommand\Shift[2]{{\cal S}{#1}.\,{#2}}
\newcommand\ShiftP[2]{(\Shift{#1}{#2})}
\newcommand\Reset[1]{\langle{#1}\rangle}

% Judgement
\newcommand\Judge[3]{{#1}\vdash{#2}:{#3}}

% CPS 変換
\newcommand\CPS[1]{[\![{#1]\!]}}


% 証明木を書く場合以下を読み込む
\usepackage{proof}

% 定理を使う場合、以下のようなものを書く
\newtheorem{definition}{定義}
\newtheorem{proposition}[definition]{命題}
\newtheorem{lemma}[definition]{補題}
\newtheorem{theorem}[definition]{定理}
% [...] 内に指定するとカウンタを共有する

%
% 以下を書き換えてタイトル部に
%------------------------------------------------------------
\title{{\gt{タイトル}}}
\author{{\gt 名前~~~~(指導教員:浅井 健一)}}
%------------------------------------------------------------

%
%
% ここから本体
%%%%%%%%%%%%%%%%%%%%%%%%%%%%%%%%%%%%%%%%%%
\begin{document}
%%%%%%%%%%%%%%%%%%%%%%%%%%%%%%%%%%%%%%%%%%
%
% 以下の3行は変更しない.
%
\raggedbottom
\maketitle
\setlength{\baselineskip}{12.5pt}

%%%%%開始ページ数を設定する(この例では1)%%%%%%%%%%%
\setcounter{page}{1}

\setlength{\abovedisplayskip}{0pt}

%------------------------------------------------------------
\section{はじめに}\label{sec:intro}
%------------------------------------------------------------
卒論の2ページの予稿のサンプルです。
最初に背景から概要を書きます。

卒論の予稿は短いので書きませんが、
PPL の論文など長い論文では以下のような全体の構成を書きます。
これが、節の番号を参照する例になっています。

本論文の構成は以下のようになっている。
まず、最初に \ref{sec:lambda} 節でラムダ式の書き方について
説明する。
\ref{sec:kajogaki} 節でいろいろな箇条書きの書き方について
説明し、
\ref{sec:code} 節ではコードの書き方の例を紹介する。
\ref{sec:bunken} 節で参考文献について触れる。
\ref{sec:theorem} 節で定理の書き方を、
\ref{sec:proof} 節で証明木の書き方を示す。
\ref{sec:conclusion} 節でまとめる。

%------------------------------------------------------------
\section{ラムダ式}\label{sec:lambda}
%------------------------------------------------------------
\texttt{macro.tex} に書いてあるような定義をあらかじめしておいて、
それを使うのが良いです。
\[
\begin{array}{lrcl}
       型 & t & := & \Int\ |\ \Arrow{t}{t} \\
       項 & e & := & x\ |\ \Lam{x}{e}\ |\ \App{e}{e}\ |\
                     \Shift{k}{e}\ |\ \Reset{e} \\
\end{array}
\]

%------------------------------------------------------------
\section{箇条書きの書き方など}\label{sec:kajogaki}
%------------------------------------------------------------
箇条書き
\begin{itemize}
       \item ひとつめ
       \item ふたつめ
       \item みっつめ
\end{itemize}
番号付き
\begin{enumerate}
       \item ひとつめ
       \item ふたつめ
       \item みっつめ
\end{enumerate}
見出し付き
\begin{description}
       \item[継続] その後の計算をさす。
       \item[限定継続] 継続のうち、その範囲が限定されているもの。
\end{description}

%------------------------------------------------------------
\section{コード}\label{sec:code}
%------------------------------------------------------------
コードを直接、書くには \texttt{verbatim} 環境が簡単です。
\begin{verbatim}
let rec fac n =
  if n = 0 then 1 else n * fac (n - 1)
\end{verbatim}
\texttt{verbatim} 環境内で tex のコマンドを使いたいときは、
\texttt{alltt} を \texttt{usepackage} して使います。
上のコードは、どうも前後の文と間がきつすぎると思うときは、
\texttt{quote} 環境に入れるというのはひとつの手です。
\begin{quote}
\begin{verbatim}
let rec fib n =
  if n < 2
  then n
  else fib (n - 1) + fib (n - 2)
\end{verbatim}
\end{quote}

これらの環境はタイプライタフォントなので、横幅をとりすぎる傾向にありま
す。なれてきたら、よりきれいにコードを書く方法を習得するのが良いかも知
れません。

%------------------------------------------------------------
\section{参考文献}\label{sec:bunken}
%------------------------------------------------------------
bibtex を使うのが良いでしょう。
\texttt{paper.bib} に型デバッガ \cite{TA2013} 関係の文献と
限定継続 \cite{DF1989, DF1990} 関係の文献を入れておきました。
また、本の例として
アルゴリズミックデバッギング \cite{Shapiro1983} も入れました。

%------------------------------------------------------------
\section{定理}\label{sec:theorem}
%------------------------------------------------------------
\begin{definition}[CPS 変換 \cite{Plotkin1975}]
\upshape % これを書くと定理の中身がイタリックではなくなる。
項 $e$ の CPS 変換 $\CPS{e}$ は以下のように定義される。
(中身は省略。)
\end{definition}

\begin{proposition}
$e$ に型がつくなら、その部分式にも型がつく。
\end{proposition}

\begin{lemma}[代入補題 \cite{WF1994}]
$\Judge{x:t_1}{e}{t}$ かつ $\Judge{}{v}{t_1}$ なら
$\Judge{}{e[v/x]}{t}$ が成り立つ。
\end{lemma}

\begin{theorem}
$e$ に型がついたら、$e$ の実行中に型エラーは起きない。
\end{theorem}

%------------------------------------------------------------
\section{証明木}\label{sec:proof}
%------------------------------------------------------------
証明木の例です。judgement もマクロとして定義するのが良いです。
\[
\begin{array}{c}
\infer[(\textsf{TVar})]
      {\Judge{\Gamma}{x}{t}}
      {\Gamma(x)=t}
\qquad
\infer[(\textsf{TLam})]
      {\Judge{\Gamma}{\Lam{x}{e}}{\Arrow{t_1}{t_2}}}
      {\Judge{\Gamma,x:t_1}{e}{t_2}} \\
\ \\
\infer[(\textsf{TApp})]
      {\Judge{\Gamma}{\App{e_1}{e_2}}{t_1}}
      {\Judge{\Gamma}{e_1}{\Arrow{t_2}{t_1}}
      &\Judge{\Gamma}{e_2}{t_2}} \\
\end{array}
\]

%------------------------------------------------------------
\section{まとめ}\label{sec:conclusion}
%------------------------------------------------------------
最後に、まとめと今後の課題などを書きます。
このサンプルは1ページで終わっていますが、必ず2ページを埋めます。
(2ページはすぐ埋まります。むしろすぐ足りなくなります。紙面が足りない
場合、下のように参考文献は多少、小さくしても構いません。)

\small % 紙面が足りない場合、参考文献は多少、小さくても良いことにする。
\bibliographystyle{jplain}
\bibliography{paper}

%%%%%%%%%%%%%%%%%%%%%%%%%%%%%%%%%%%%
\end{document}
%%%%%%%%%%%%%%%%%%%%%%%%%%%%%%%%%%%%
