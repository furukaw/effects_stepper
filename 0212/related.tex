\section{関連研究}
\label{section:related}

ステッパはもともと Racket に対して作られた。
これは Clements ら \cite{clements01} が設計したもので、
スタックに continuation mark と呼ばれるマークを付けることで現在の評価
文脈を再構成できるようにしている。しかし、例外処理などの構文には対応し
ていない。我々は引数にコンテキストの情報を渡すことで、OCaml に対するス
テッパを設計した \cite{FCA19}。このステッパは OCaml の例外処理にも対応
している。
以上のステッパはいずれも big-step のインタプリタに手を加える形で作られ
ている。一方、Whitington と Ridge \cite{EPTCS294.3} は small-step のイ
ンタプリタを直接、書くことで OCaml に対するステッパを実装した。しか
し、以上のステッパはいずれも algebraic effects には対応していない。

algebraic effects に対する意味論は、これまで small-step の
意味論 \cite{10.1145/2500365.2500590}
あるいは CPS による意味論 \cite{e6cb0c3222794e48bf38cf44e46fe4aa} が
与えられて来た。しかし、後者は入力言語が A-正規形であることを仮定して
いるのに加え、継続がフレームのリストで与えられており、通常の CPS イン
タプリタにはなっていない。本論文で与えた CPS インタプリタは、入力言語
を制限しておらず、また継続も普通の一引数関数となっている。

上記以外の algebraic effects に関する研究としては、ハンドラの挙動が異
なる shallow ハンドラの研究 \cite{10.1007/978-3-030-02768-1_22} や
algebraic effects を含むプログラムに関する論理関係を
定義する研究 \cite{10.1145/3158096} などがあげられる。本論文で扱ってい
るハンドラは従来の deep ハンドラである。shallow ハンドラにも対応できると
考えているが、これは今後の課題である。

%\cite{10.1145/2500365.2500590}
%\cite{10.1007/978-3-030-02768-1_22}
%\cite{e6cb0c3222794e48bf38cf44e46fe4aa}
%\cite{10.1145/3158096}
%\cite{PRETNAR201519}
