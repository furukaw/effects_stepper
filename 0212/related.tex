\section{関連研究}
\label{section:related}

ステッパはもともと Racket に対して作られた。
これは Clements ら \cite{clements01} が設計したもので、
スタックに continuation mark と呼ばれるマークを付けることで現在の評価
文脈を再構成できるようにしている。しかし、例外処理などの構文には対応し
ていない。我々は引数にコンテキストの情報を渡すことで、OCaml に対するス
テッパを設計した \cite{FCA19}。このステッパは OCaml の例外処理にも対応
している。
以上のステッパはいずれも big step のインタプリタに手を加える形で作られ
ている。一方、Whitington と Ridge \cite{EPTCS294.3} は small step のイ
ンタプリタを直接、書くことで OCaml に対するステッパを実装した。しか
し、以上のステッパはいずれも algebraic effect handlers には対応していない。

Danvy ら \cite{10.1145/1411204.1411206} は
big step インタプリタにCPS変換と非関数化を施すことで抽象機械が導出できることを
示した。本研究では同様の方法によってコンテキストの情報を得たが、
algebraic effect handlers は限定継続を扱うため継続が区切られており、
\texttt{handle} で区切られた内部の継続と全体の継続それぞれについて変換をする必要があったため、2度この変換を行った
(\texttt{handle} の内部については、もともと CPS になっているインタプリタを用いたためCPS変換は不要だった)。

我々のグループでは以前、shift/reset の入った体系に対して CPS 変換と非
関数化をかけることで、shift/reset に対する仮想機械を得た \cite{AK2010}。この論文
は、これを algebraic effect handlers に対して行ったものである。その
際、最初の big step のインタプリタを作るところは非自明だったが、そのあ
との変換は以前の研究の通りに行うことができた。

algebraic effect handlers に対する意味論は、これまで small step の
意味論 \cite{10.1145/2500365.2500590}
あるいは CPS による意味論 \cite{10.1145/2976022.2976033, e6cb0c3222794e48bf38cf44e46fe4aa} が
与えられてきた。
しかし、後者は、完全には末尾呼び出しにはなっていない 2CPS のインタプリ
タと継続が非関数化されたインタプリタしか示されていない。本論文で与えた
CPS インタプリタは、高階の継続を直接使っており、さらに CPS 変換をかけ
れば全てが末尾呼び出しになっている 2CPS のインタプリタとなる。

上記以外の algebraic effect handlers に関する研究としては、ハンドラの挙動が異
なる shallow ハンドラの研究 \cite{10.1007/978-3-030-02768-1_22}
などがあげられる。本論文で扱ってい
るハンドラは従来の deep ハンドラである。shallow ハンドラにも対応できると
考えているが、これは今後の課題である。

