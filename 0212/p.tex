\documentclass[japanese,draft]{jssst_ppl} %% 日本語 (default)
% \documentclass[english]{jssst_ppl} %% English
% \documentclass[japanese,draft]{jssst_ppl} %% You can use the draft option

\usepackage{latexsym}
\usepackage{alltt}
\usepackage[dvipdfmx]{graphicx}
\usepackage[dvipdfmx]{color}
\usepackage{proof}
\usepackage{setspace}
\usepackage{my-tt}

\definecolor{lightgreen}{rgb}{0.71, 0.93, 0.71}
\definecolor{purple}{rgb}{0.80, 0.60, 1.00}
\definecolor{lg}{rgb}{0.9, 0.9, 0.9}

% 型
\newcommand\Int{\textsf{int}}
\newcommand\Arrow[2]{{#1}\rightarrow{#2}}

% ラムダ式
\newcommand\Lam[2]{\lambda{#1}.\,{#2}}
\newcommand\LamP[2]{(\Lam{#1}{#2})}
\newcommand\App[2]{{#1}\,{#2}}
\newcommand\AppP[2]{(\App{#1}{#2})}
\newcommand\Shift[2]{{\cal S}{#1}.\,{#2}}
\newcommand\ShiftP[2]{(\Shift{#1}{#2})}
\newcommand\Reset[1]{\langle{#1}\rangle}

% Judgement
\newcommand\Judge[3]{{#1}\vdash{#2}:{#3}}

% CPS 変換
\newcommand\CPS[1]{[\![{#1]\!]}}

% 型無し big-step 実行規則
\newcommand\Eval[2]{\texttt{{#1}}\Downarrow\texttt{{#2}}}


\title{algebraic effect handlers を含むプログラムのステップ実行}
\author{古川 つきの, 浅井 健一}
\inst{
  お茶の水女子大学\\
\texttt{furukawa.tsukino@is.ocha.ac.jp, asai@is.ocha.ac.jp}
}
\begin{document}
\maketitle
\begin{abstract}
  ステッパはプログラムの実行過程を見せるツールである。
  これまで様々な言語に対するステッパが作られてきたが、
  shift/reset や algebraic effect handlers
  といった継続を明示的に扱う言語機能をサポートするステッパは作られていない。
  継続を扱うプログラムの挙動を理解するのは困難なので、
  そういった言語に対応したステッパがあると役に立つと思われる。
  それを作ることが本研究の目的である。

  ステッパは簡約のたびにその時点でのプログラム全体を出力するインタプリタなので、
  実行している部分式のコンテキストの情報が常に必要になる。
  継続を扱うような複雑な機能を持つ言語を対象にしたステッパでは、
  コンテキストがどのような構造をしているかが自明でない。
  そこで、通常のインタプリタ関数にプログラム変換を施すことで機械的にコンテキストの情報を保持させたステッパを実装する方法を示し、
  実際に型無しλ計算と algebraic effect handlers から成る言語に対するステッパを実装する。
\end{abstract}

\section{はじめに}

プログラムの実行の様子を確認する方法に、実行が1段階進むごとにどのよう
な状態になっているのかを書き連ねる方法がある。このツールは、代数的ス
テッパ(以後、単にステッパと書く)と呼ばれ、プログラムが代数的に書き換
わる様子を1ステップずつ表示してくれるものである。
例えば OCaml のプログラム \texttt{let a = 1 + 2 in 4 + a} を入力される
と、ステッパは以下のような実行ステップを表す文字列を出力する。

\begin{verbatim}
Step 0:  (let a = (1 + 2) in (4 + a))
Step 1:  (let a = 3 in (4 + a))

Step 1:  (let a = 3 in (4 + a))
Step 2:  (4 + 3)

Step 2:  (4 + 3)
Step 3:  7
\end{verbatim}

このように1段階ずつ確認していけば、具体的にどのような計算がされるのか
を観察することができ、プログラムの動きを理解しやすい。またこの方法はデ
バッグにおいても有用で、どの段階で想定と違うことが起こっているのかが見
えるので、プログラムのどの部分がその原因なのかが分かりやすくなる。その
ため、お茶の水女子大学では実際に関数型言語の授業でステッパを本格的に使
用している。

しかし、継続を操作するようなプログラムだった場合、ステッパをどのように
作ったら良いのかは明らかではない。そのような複雑なプログラムでこそ実行
の様子を詳細に追いたいところだが、現在のところそのような言語に対するス
テッパは作られていない。
わずかに、我々が過去の研究 \cite{FCA19} で例外処理のための
構文 try-with を含む言語についてのステッパを実装した程度である。

そこで我々は、ステッパを、継続を明示的に扱う言語機能に対応させることを
目指している。ステッパは、通常のインタプリタ関数に出力機能を追加するこ
とで実装できるが、その際、式全体を再構成するためにコンテキストの情報が
必要になる。以前の研究 \cite{FCA19} では、部分式の簡約に進むたびにコン
テキストを表すデータ型を作成していたが、継続を操作するようなプログラム
の場合、どのようなコンテキストにすれば良いのかは即座には明らかではな
い。

そこで、本論文では、インタプリタに対して CPS 変換 \cite{PLOTKIN1975125} と
非関数化 \cite{Reynolds1998} を施すことで機械的にコンテキストの情報を
得る。この方法を使うと、継続を操作するような言語でも機械的にコンテキス
トの情報を得ることができ、それを使ってステッパを作ることができるように
なる。本論文では、この手法を algebraic effects を含む言語に対して適用
し、algebraic effects を含む言語に対するステッパを作成する。また、その
過程で algebraic effects を含む言語に対する definitional interpreter
を示す。本論文では詳しくは述べないが、この手法は shift/reset に対する
ステッパの作成にも使うことができる。

本論文の構成は以下の通りである。
まず \ref{section:context} 節でステッパを実装する方法を紹介する。
そして \ref{section:definition} 節で algebraic effects を含む言語およびそのインタプリタを定義し、
\ref{section:transform} 節でインタプリタを変換してステッパを得るまでの過程を説明する。
\ref{section:languages} 節では他のいくつかの言語に対するステッパを同様の変換によって得ることについて議論する。
\ref{section:related} 節で関連研究について触れ、\ref{section:conclusion} 節でまとめる。

%継続を操作するプログラムの挙動を理解するためには、プログラムがどのように実行されていくのかを詳細に追う必要がある。
%その継続を得た時点での実行の状態が分からなければどのような継続なのかが分からないからである。

%そのようなプログラムの実行の様子を確認する方法に、実行が1段階進むごとにどのような状態になっているのかを書き連ねる方法がある。1段階ずつ確認していけば、具体的にどのような計算がされるのかを観察することができ、プログラムの動きを理解しやすい。またこの方法はデバッグにおいても有用で、どの段階で想定と違うことが起こっているのかが見えるので、プログラムのどの部分がその原因なのかが分かりやすくなる。

%そこで我々は、プログラムが代数的に書き換わる様子を1ステップずつ表示するツールである代数的ステッパを、継続を明示的に扱う言語機能に対応させることを目指している。
%ステッパは、例えば OCaml のプログラム \texttt{let a = 1 + 2 in 4 + a} を入力されると、以下のような実行ステップを表す文字列を出力する。

%\begin{quote}
%\begin{verbatim}
%Step 0:  (let a = (1 + 2) in (4 + a))
%Step 1:  (let a = 3 in (4 + a))
%
%Step 1:  (let a = 3 in (4 + a))
%Step 2:  (4 + 3)
%
%Step 2:  (4 + 3)
%Step 3:  7
%\end{verbatim}
%\end{quote}

%ステッパはプログラムを評価するプログラムなのでインタプリタの一種である。古川ら(2019)\cite{FCA19}は、例外処理のための構文 try-with を含む言語について、通常のインタプリタ関数に出力機能を追加することでステッパを実装した。インタプリタは部分式を再帰的に実行するが、その先で簡約があった場合に式全体を出力するために、部分式を再帰的に実行している時にそのコンテキストの情報を得ることが必要になる。古川ら(2019)\cite{FCA19}は評価順序をもとにコンテキストの構造を考えてコンテキストを表すデータ型を定義していたが、本研究ではインタプリタ関数に CPS 変換 \cite{PLOTKIN1975125} および非関数化 \cite{Reynolds1998} という変換を施すことでコンテキストの情報を引数に持つインタプリタ関数を導出し、それを用いてプログラム全体を再構成して出力する方法を提案する。そして実際に algebraic effects を含む言語についてインタプリタ関数を定義し、それを変換することによってステッパ関数を導出する過程について説明する。


%\section{ステッパの実装におけるコンテキスト}
\section{ステッパの実装方法とコンテキスト}
\label{section:context}

ステッパは small-step による実行と同じなので、small-step のインタプリ
タを書けば実装できる。実際、Whitington \& Ridge \cite{EPTCS294.3} は small-step のインタプリ
タを書くことで OCaml に対するステッパを実装している。
しかし、small-step のインタプリタをメンテナンスするのは簡単ではない。
また、ステップ実行中に関数呼び出し単位でスキップする機能をつけようと思
うとインタプリタは big-step で書かれていた方が都合が良い。そこで、我々
の過去の研究 \cite{FCA19} では、big-step のインタプリタを元にして
ステッパを作成している。ここでも、そのアプローチをとる。

図 \ref{figure:lambda} に OCaml による型無し$\lambda$計算の定義と代入
ベースの big-step インタプリタの実装を示す。
関数 \texttt{subst :\ e -> (string * v) list -> e} は代入関数であり、
\texttt{subst e [(x, v)]} は式 \texttt{e} の中の
全ての変数 \texttt{x} を値 \texttt{v} に置換した式を返す。

%ステッパはインタプリタにステップ出力機能が加わったものなので、通常のインタプリタ関数に書き加えることで実装できる。図 \ref{figure:lambda} に OCaml による型無し$\lambda$計算の定義とインタプリタの実装を示す。関数 \texttt{subst : e -> (string * v) list -> e} は代入関数であり、\texttt{subst e [(x, v)]} は式 \texttt{e} の中の全ての変数 \texttt{x} を値 \texttt{v} に置換した式を返す。

%ステッパは small-step での実行過程を見せるものであるが、small-step でなく big-step のインタプリタを基にして作ることで、関数適用や配列に対する処理やループ等のひとまとまりの実行の開始と終了を検知して「その部分の最後のステップまで飛ばす」等の機能を作ることが容易になる\ref{FCA19}ため、big-step のインタプリタを利用している。

\begin{figure}
\begin{verbatim}
(* 値 *)
type v = Var of string      (* x *)
       | Fun of string * e  (* fun x -> e *)

(* 式 *)
and e = Val of v      (* 値 *)
       | App of e * e  (* e e *)

(* インタプリタ *)
let rec eval (exp : e) : v = match exp with
  | Val (v) -> v  (* 値ならそのまま返す *)
  | App (e1, e2) ->
    let v2 = eval e2 in  (* 引数部分を実行 *)
    let v1 = eval e1 in  (* 関数部分を実行 *)
    let reduct = match v1 with
      | Fun (x, e) -> subst e [(x, v2)]  (* 代入 e[v2/x] *)
      | _ -> failwith "type error" in  (* 関数部分が関数でなければ型エラー *)
    eval reduct  (* 代入後の式を実行 *)
\end{verbatim}
\caption{型無し$\lambda$計算とそのインタプリタ}
\label{figure:lambda}
\end{figure}

このインタプリタをステッパにするには、簡約をする際に簡約前後のプログラ
ムを出力する機能を追加すればよい。
しかしステッパが出力したいのは実行中の部分式ではなく式全体であり、コン
テキストを含めた式全体を出力するためには、実行中の式の構文木の他にコン
テキストの情報が必要である。

%このインタプリタをステッパにするには、簡約をする際に簡約前後のプログラムを出力する機能を追加すればよい。
%関数 \texttt{eval} は実行する式の部分式を再帰的に実行する。すると、再帰呼び出しされた \texttt{eval} の中では、引数に実行中の部分式の情報しか与えられていない。
%しかしステッパが出力したいのは実行中の部分式ではなく式全体であり、コンテキストを含めた式全体を出力するためには、実行中の式の構文木の他にコンテキストの情報が必要である。

コンテキストの情報を得るために、Clements ら\cite{clements01} は Racket
の continuation mark を使用してコンテキストフレームの情報を記録するこ
とでステッパを実装した。本研究ではそのような特殊な機能は使わずに、
インタプリタ関数に明示的にコンテキスト情報のための引数を追加する。
図\ref{figure:lambda}のインタプリタに
その変更を施すと、図\ref{figure:lambda_stepper}のようになる。ここで、
関数 \texttt{memo :\ e -> e -> c -> unit} は、簡約前の式、簡約後の式、
コンテキスト情報の3つを引数にとり、コンテキスト情報を利用して簡約前後
の式全体をそれぞれ出力するものである。

%コンテキストの情報を得るために、Clements ら\cite{clements01} は Racket の continuation mark を使用してコンテキストフレームの情報を記録したが、本研究ではインタプリタ関数に明示的にコンテキスト情報のための引数を追加する。筆者らは以前\cite{FCA19}、実際にインタプリタに引数を追加することによってステッパを実装した。図\ref{figure:lambda}のインタプリタにその変更を施すと、図\ref{figure:lambda_stepper}のようになる。ここでは関数 \texttt{memo :\ e -> e -> c -> unit} は、簡約前の式、簡約後の式、コンテキスト情報の3つの引数をとり、コンテキスト情報を利用して簡約前後の式にそれぞれコンテキストを付加することで簡約前後の式全体を得て出力する。

\begin{figure}
\begin{verbatim}
(* コンテキスト *)
type c = CId             (* [.] *)
       | CApp2 of e * c  (* [e [.]] *)
       | CApp1 of v * c  (* [[.] v] *)

(* 出力しながら再帰的に実行 *)
let rec eval (exp : e) (c : c) : v = match exp with
  | Val (v) -> v
  | App (e1, e2) ->
    let v2 = eval e2 (CApp2 (e1, c)) in  (* コンテキストを1層深くする *)
    let v1 = eval e1 (CApp1 (v2, c)) in  (* コンテキストを1層深くする *)
    let redex = App (Val v1, Val v2) in
    let reduct = match v1 with
      | Fun (x, e) -> subst e [(x, v2)]
      | _ -> failwith "type error" in
    memo redex reduct c;  (* コンテキストを利用して式全体を出力 *)
    eval reduct c

(* 実行を始める *)
let stepper (exp : e) = eval exp CId
\end{verbatim}
\caption{型無し$\lambda$計算に対するステッパ}
\label{figure:lambda_stepper}
\end{figure}

図\ref{figure:lambda_stepper}のように、コンテキストを表すデータ型を定
義して再帰呼び出し時の構造に合わせて引数として渡すようにすれば、式全体
を再構成して出力することが可能になる。ここで、コンテキストを表すデータ
型は、評価文脈そのものになっていることに気がつく。評価文脈のデータ型は、
big-step のインタプリタを CPS 変換し、非関数化すると機械的に得られるこ
とが知られている。これは、我々が手動で定義したコンテキストのデータは、
機械的に導出できることを示唆している。

$\lambda$計算に対するステッパであれば、手動でコンテキストの型を定義す
るのは簡単だが、言語が複雑になってくると必ずしもこれは自明ではない。実
際、以前の研究 \cite{FCA19} で try-with 構文を含む言語のステッパを実装
したときには、コンテキストを try-with 構文で区切る必要があったため、
コンテキストの構造が一次元的でなく、リストのリストになった。
algebraic effects などが入った場合、どのようなコンテキストを使えば良い
のかはまた別途、考慮する必要がある。このような場合、機械的にコンテキス
トの定義を導出できることにはメリットがある。
次節以降ではそのような方針で algebraic effects に対するステッパを導出
する。

%図\ref{figure:lambda_stepper}のように、コンテキストを表すデータ型を定義して再帰呼び出し時の構造に合わせて引数として渡すようにすれば、式全体を再構成して出力することが可能になる。以前の研究 \cite{FCA19} ではこの方法を用いて try-with 構文を含む言語のステッパを実装することに成功した。
%しかし、try-with のような制御構文を含む言語では、コンテキストを区切ってある区間のコンテキストを一度に捨てるなどの操作が必要になるため、コンテキストの構造が一次元的でなくなり、複雑なデータ構造の定義が必要になる可能性がある \cite{FCA19} 。

%ところが、図\ref{figure:lambda_stepper}の\texttt{c}型の定義を見ると、各コンストラクタはインタプリタ関数の「どの再帰呼び出しか」に対応している。コンテキストの型は評価順序によって定まるものであり、評価順序はインタプリタ関数で定義されているので、コンテキストを表すデータ型の定義はインタプリタ関数から導出できるものであると考えられる。次節以降ではその導出方法の1つを提案する。


\section{言語とインタプリタの定義}
\label{section:definition}

この節では、型無し$\lambda$計算と algebraic effects からなる言語とそのインタプリタを定義する。

\subsection{対象言語の構文}
\label{subsection:syntax}

対象言語の OCaml による定義を図 \ref{figure:syntax} に示す。
値のコンストラクタ \texttt{Cont} は継続を表すコンストラクタであり、入力プログラムに含まれることはないが、ステップ実行の過程で現れる。
式の型 \texttt{e} で表されるものが対象言語のプログラムである。

\begin{figure}
\begin{verbatim}
(* 値 *)
type v = Var of string              (* x *)
       | Fun of string * e          (* fun x -> e *)
       | Cont of string * (k -> k)  (* 継続 fun x => ... *)
(* ハンドラ *)
and h = {
  return : string * e;                       (* handler {return x -> e,      *)
  ops : (string * string * string * e) list  (*          op(x; k) -> e, ...} *)
}
(* 式 *)
and e = Val of v          (* v *)
      | App of e * e      (* e e *)
      | Op of string * e  (* op e *)
      | With of h * e     (* with h handle e *)
(* 継続 *)
and k = v -> a
(* 実行結果 *)
and a = Return of v               (* 正常終了 *)
      | OpCall of string * v * k  (* オペレーション呼び出し *)
\end{verbatim}
\caption{対象言語の定義}
\label{figure:syntax}
\end{figure}

\subsection{CPS インタプリタ}

図 \ref{figure:syntax} の言語に対する、call-by-value かつ right-to-left のインタプリタを図 \ref{figure:1cps} に定義する。ただし、関数 \texttt{subst :\ e -> (string * v) list -> e} は代入のための関数であり、\texttt{subst e [(x, v); (k, cont\_value)]} は \texttt{e} の中の変数 \texttt{x} と変数 \texttt{k} に同時にそれぞれ値 \texttt{v} と値 \texttt{cont\_value} を代入した式を返す。関数 \texttt{search\_op} はハンドラ内のオペレーションを検索する関数で、例えば \texttt{handler \{return x -> x, op1(y, k) -> k y\}} を表すデータを \texttt{h} とすると \texttt{search\_op "op2" h} は \texttt{None} を返し \texttt{search\_op "op1" h} は \texttt{Some (y, k, App (Var "k", Var "y"))} を返す。

\begin{figure}
\begin{verbatim}
(* インタプリタ *)
let rec eval (exp : e) (k : k) : a = match exp with
  | Val (v) -> k v  (* 継続に値を渡す *)
  | App (e1, e2) ->
    eval e2 (fun v2 ->
        eval e1 (fun v1 -> match v1 with
            | Fun (x, e) ->
              let reduct = subst e [(x, v2)] in
              eval reduct k
            | Cont (k') ->
              (k' k) v2  (* 現在の継続と継続値が保持するメタ継続を合成して値を渡す *)
            | _ -> failwith "type error"))
  | Op (name, e) ->
    eval e (fun v -> OpCall (name, v, k))
  | With (h, e) ->
    let a = eval e (fun v -> Return v) in
    apply_handler k h a

(* ハンドラを処理する関数 *)
and apply_handler (k : k) (h : h) (a : a) : a = match a with
  | Return v ->                          (* handle 節内が値 v を返したとき *)
    (match h with {return = (x, e)} ->   (* handler {return x -> e, ...} *)
       let reduct = subst e [(x, v)] in  (* e[v/x] に簡約される *)
       eval reduct k)                    (* e[v/x] を実行 *)
  | OpCall (name, v, k') ->              (* オペレーション呼び出し *)
    (match search_op name h with
     | None ->                           (* ハンドラで定義されていない場合 *)
       OpCall (name, v, (fun v ->        (* OpCall の継続に現在の継続を合成 *)
           let a' = k' v in
           apply_handler k h a'))
     | Some (x, y, e) ->                 (* ハンドラで定義されている場合 *)
       let cont_value =
         Cont (fun k'' -> fun v ->       (* 適用時にその後の継続を受け取って合成 *)
                 let a' = k' v in
                 apply_handler k'' h a') in
       let reduct = subst e [(x, v); (y, cont_value)] in
       eval reduct k)

(* 初期継続を渡して実行を始める *)
let interpreter (e : e) : a = eval e (fun v -> Return v)
\end{verbatim}
\caption{継続渡し形式で書かれたインタプリタ}
\label{figure:1cps}
\end{figure}


\section{インタプリタの変換}
\label{section:transform}

本節では、\ref{section:definition}節で定義したインタプリタ(図\ref{figure:1cps})を変換することで、コンテキストの情報を保持するインタプリタを得る方法を示す。用いるプログラム変換は非関数化とCPS変換の2種類である。これらの変換はプログラムの動作を変えないので、変換の結果得られるインタプリタと図\ref{figure:1cps}のインタプリタは、同じ引数 \texttt{e} に対して同じ値を返す。

\subsection{非関数化}
\label{section:2defun}

まず、図\ref{figure:1cps}のプログラムを非関数化する。具体的には以下を施す。
\begin{enumerate}
\item \texttt{k} 型すなわち \texttt{v -> a} 型の匿名関数全てを、関数内に現れる全ての自由変数を引数に持つコンストラクタに分類し、継続の型 \texttt{k} をそれらのコンストラクタから構成されるヴァリアント型に変更する(図\ref{figure:k_2defun} のようになる)。
\item \texttt{v -> a} 型の匿名関数全てを、該当するコンストラクタに置き換える
\item 継続に引数を渡している部分 \texttt{k v} を \texttt{apply\_k k v} に置き換え、意味が変わらないように関数 \texttt{apply\_k} を定義する
\end{enumerate}

\begin{figure}
\begin{verbatim}
and k = FId
      | FApp2 of e * k
      | FApp1 of v * k
      | FOp of string * k
      | FCall of k * h * k
\end{verbatim}
\caption{非関数化後の継続の型}
\label{figure:k_2defun}
\end{figure}

変換後のプログラムを図\ref{figure:2defun}に示す。

\begin{figure}
\begin{verbatim}
let rec eval (exp : e) (k : k) : a = match exp with
  | Val (v) -> apply_in k v
  | App (e1, e2) -> eval e2 (FApp2 (e1, k))
  | Op (name, e) -> eval e (FOp (name, k))
  | With (h, e) ->
    let a = eval e FId in
    apply_handler k h a

and apply_in (k : k) (v : v) : a = match k with
  | FId -> Return v
  | FApp2 (e1, k) -> let v2 = v in
    eval e1 (FApp1 (v2, k))
  | FApp1 (v2, k) -> let v1 = v in
    (match v1 with
     | Fun (x, e) ->
       let reduct = subst e [(x, v2)] in
       eval reduct k
     | Cont (x, k') ->
       apply_in (k' k) v2
     | _ -> failwith "type error"
    )
  | FOp (name, k) -> OpCall (name, v, k)
  | FCall (k_last, h, k') ->
    let a = apply_in k' v in
    apply_handler k_last h a

and apply_handler (k_last : k) (h : h) (a : a) : a = match a with
  | Return v ->
    (match h with {return = (x, e)} ->
       let reduct = subst e [(x, v)] in
       eval reduct k_last)
  | OpCall (name, v, k') ->
    (match search_op name h with
     | None ->
       OpCall (name, v, FCall (k_last, h, k'))
     | Some (x, k, e) ->
       let new_var = gen_var_name () in
       let cont_value =
         Cont (new_var, fun k_last -> FCall (k_last, h, k')) in
       let reduct = subst e [(x, v); (k, cont_value)] in
       eval reduct k_last)

let stepper (e : e) : a = eval e FId
\end{verbatim}
\caption{非関数化した CPS インタプリタ}
\label{figure:2defun}
\end{figure}

非関数化したことで継続 \texttt{k} がコンストラクタとして表されるようになったので、継続の構造を参照することや、継続を部分的に書き換えることが可能になった。
具体的な \texttt{k} の構造の例を示す。図 \ref{figure:2defun} の関数 \texttt{stepper} に入力プログラム \texttt{((fun a -> a) (with handler \{return x -> x, a(x; k) -> x\} handle ((fun b -> b) (a (fun c -> c))) (fun d -> d)))} を表す構文木を渡して実行を始めた場合、\texttt{(a (fun c -> c))} を関数 \texttt{eval} に渡して実行を始める際の継続は \texttt{FApp2 (Fun ("b", Var "b"), FApp1 (Fun ("d", Var "d")))} である。これは式 \texttt{(a (fun c -> c))} のコンテキスト \texttt{((fun a -> a) (with handler \{return x -> x, a(x; k) -> x\} handle ((fun b -> b) [.]) (fun d -> d)))} のうち、\texttt{handle} の内側に対応している。\texttt{handle} から外側が継続に含まれないのは、関数 \texttt{eval} で \texttt{with h handle e} の \texttt{e} の実行の再帰呼び出し時に初期継続を表す \texttt{FId} を渡しているためである。コンテキスト全体に対応した継続を得るために、この後の変換をさらに施す。

\subsection{CPS 変換}
\label{subsection:3cps}

図\ref{figure:2defun} では、末尾再帰でない再帰呼び出しの際に継続が初期化されてしまうせいでコンテキスト全体に対応する情報が継続に含まれなかったので、全ての継続を引数に持つようにするため、さらに CPS 変換を施す。この変換によって現れる継続は \texttt{a -> a} 型である。この型 \texttt{a -> a} の名前を \texttt{k2} とする。
変換したプログラムが図 \ref{figure:3cps} である。

\begin{figure}
\begin{verbatim}
let rec eval (exp : e) (k : k) (k2 : k2) : a = match exp with
  | Val (v) -> apply_in k v k2
  | App (e1, e2) -> eval e2 (FApp2 (e1, k)) k2
  | Op (name, e) -> eval e (FOp (name, k)) k2
  | With (h, e) -> eval e FId (fun a -> apply_handler k h a k2)

and apply_in (k : k) (v : v) (k2 : k2) : a = match k with
  | FId -> k2 (Return v)
  | FApp2 (e1, k) -> let v2 = v in
    eval e1 (FApp1 (v2, k)) k2
  | FApp1 (v2, k) -> let v1 = v in
    (match v1 with
     | Fun (x, e) ->
       let reduct = subst e [(x, v2)] in
       eval reduct k k2
     | Cont (x, k') ->
       apply_in (k' k) v2 k2
     | _ -> failwith "type error")
  | FOp (name, k) -> k2 (OpCall (name, v, k))
  | FCall (k_last, h, k') ->
    apply_in k' v (fun a -> apply_handler k_last h a k2)

and apply_handler (k_last : k) (h : h) (a : a) (k2 : k2) : a = match a with
  | Return v ->
    (match h with {return = (x, e)} ->
       let reduct = subst e [(x, v)] in
       eval reduct k_last k2)
  | OpCall (name, v, k') ->
    (match search_op name h with
     | None ->
       k2 (OpCall (name, v, FCall (k_last, h, k')))
     | Some (x, k, e) ->
       let new_var = gen_var_name () in
       let cont_value =
         Cont (new_var, fun k_last -> FCall (k_last, h, k')) in
       let reduct = subst e [(x, v); (k, cont_value)] in
       eval reduct k_last k2)

let stepper (e : e) : a = eval e FId (fun a -> a)
\end{verbatim}
\caption{CPS 変換した非関数化した CPS インタプリタ}
\label{figure:3cps}
\end{figure}

\subsection{非関数化}
\label{subsection:4defun}

CPS 変換したことにより新たに現れた \texttt{a -> a} 型の匿名関数を非関数化する。非関数化によって型 \texttt{k2} の定義は図 \ref{figure:k2_4defun} に、ステッパ関数は図 \ref{figure:4defun} に変換される。

\begin{figure}
\begin{verbatim}
type k2 = GId
        | GHandle of k * h * k2
\end{verbatim}
\caption{2回目の非関数化後の継続の型}
\label{figure:k2_4defun}
\end{figure}

\begin{figure}
\begin{verbatim}
let rec eval (exp : e) (k : k) (k2 : k2) : a = match exp with
  | Val (v) -> apply_in k v k2
  | App (e1, e2) -> eval e2 (FApp2 (e1, k)) k2
  | Op (name, e) -> eval e (FOp (name, k)) k2
  | With (h, e) -> eval e FId (GHandle (k, h, k2))

and apply_in (k : k) (v : v) (k2 : k2) : a = match k with
  | FId -> apply_out k2 (Return v)
  | FApp2 (e1, k) -> let v2 = v in
    eval e1 (FApp1 (v2, k)) k2
  | FApp1 (v2, k) -> let v1 = v in
    (match v1 with
     | Fun (x, e) ->
       let reduct = subst e [(x, v2)] in
       eval reduct k k2
     | Cont (x, k') ->
       apply_in (k' k) v2 k2
     | _ -> failwith "type error")
  | FOp (name, k) -> apply_out k2 (OpCall (name, v, k))
  | FCall (k_last, h, k') ->
    apply_in k' v (GHandle (k_last, h, k2))

and apply_out (k2 : k2) (a : a) : a = match k2 with
  | GId -> a
  | GHandle (k, h, k2) -> apply_handler k h a k2

and apply_handler (k_last : k) (h : h) (a : a) (k2 : k2) : a = match a with
  | Return v ->
    (match h with {return = (x, e)} ->
       let reduct = subst e [(x, v)] in
       eval reduct k_last k2)
  | OpCall (name, v, k') ->
    (match search_op name h with
     | None ->
       apply_out k2 (OpCall (name, v, FCall (k_last, h, k')))
     | Some (x, k, e) ->
       let new_var = gen_var_name () in
       let cont_value =
         Cont (new_var, fun k_last -> FCall (k_last, h, k')) in
       let reduct = subst e [(x, v); (k, cont_value)] in
       eval reduct k_last k2)

let stepper (e : e) : a = eval e FId GId
\end{verbatim}
\caption{非関数化して CPS 変換して非関数化した CPS インタプリタ}
\label{figure:4defun}
\end{figure}

この非関数化によって、引数 \texttt{k} と引数 \texttt{k2} からコンテキスト全体の情報が得られるようになった。
\ref{subsection:2defun} 節で示した例について比較する。\texttt{stepper} に入力プログラム \texttt{((fun a -> a) (with handler \{return x -> x, a(x; k) -> x\} handle ((fun b -> b) (a (fun c -> c))) (fun d -> d)))} を表す構文木を渡して実行を始めた場合、\texttt{(a (fun c -> c))} を表す構文木を関数 \texttt{eval} に渡して実行を始める際の継続 \texttt{k} は \ref{subsection:2defun} 節と同様に \texttt{{FApp2 (Fun ("b", Var "b"), FApp1 (Fun ("d", Var "d")))}} である。そして継続 \texttt{k2} は \texttt{GHandle ()}
具体的な \texttt{k} の構造の例を示す。図 \ref{figure:2defun} の関数 \texttt{stepper} に入力プログラム \texttt{((fun a -> a) (with handler \{return x -> x, a(x; k) -> x\} handle ((fun b -> b) (a (fun c -> c))) (fun d -> d)))} を表す構文木を渡して実行を始めた場合、\texttt{(a (fun c -> c))} を関数 \texttt{eval} に渡して実行を始める際の継続は \texttt{FApp2 (Fun ("b", Var "b"), FApp1 (Fun ("d", Var "d")))} である。これは式 \texttt{(a (fun c -> c))} のコンテキスト \texttt{((fun a -> a) (with handler \{return x -> x, a(x; k) -> x\} handle ((fun b -> b) [.]) (fun d -> d)))} のうち、\texttt{handle} の内側に対応している。\texttt{handle} から外側が継続に含まれないのは、関数 \texttt{eval} で \texttt{with h handle e} の \texttt{e} の実行の再帰呼び出し時に初期継続を表す \texttt{FId} を渡しているためである。コンテキスト全体に対応した継続を得るために、この後の変換をさらに施す。


\section{他の言語への対応}
\label{section:languages}

\ref{section:definition} 節で示した algebraic effects を含む言語の CPS インタプリタをステッパにするには、非関数化、CPS変換、非関数化が必要だったが、他のいくつかの言語についても同様にインタプリタを変換することでステッパを導出することを試みた。それぞれの言語のステッパ導出について説明する。


\subsection{型無し$\lambda$計算}
\label{subsection:lambda}

型無し$\lambda$計算の DS インタプリタは、CPS変換して非関数化したら全てのコンテキストを引数に保持するインタプリタになり、出力関数を入れるのみでステッパを作ることができた。これは、継続を区切って一部を捨てたり束縛したりするという操作が無いためである。


\subsection{try-with}
\label{subsection:try_with}

try-with は、algebraic effects が限定継続を変数に束縛するのと違って、例外が起こされたときに限定継続を捨てるという機能である。よって継続を表す値は現れないので、インタプリタを CPS で書く必要は無い。Direct style でインタプリタを書いた場合、最初にCPS変換をすることで、CPSインタプリタと同様の変換によってステッパが導出できた。最初からCPSインタプリタを書いていれば \ref{section:transform}節と同様の手順になる。


\subsection{shift/reset}
\label{subsection:shift/reset}

shift/reset は algebraic effects と同様に限定継続を変数に束縛して利用することができる機能である。\ref{section:transform} 節で行ったのと全く同様に、CPS インタプリタを非関数化、CPS変換、非関数化したらコンテキストが表れ、ステッパが得られた。


\subsection{Multicore OCaml}
\label{subsection:multicore_ocaml}

Multicore OCaml は、OCaml の構文に algebraic effects を追加した構文を持つ。我々は \ref{section:transform} 節で得られたステッパをもとにして、Multicore OCaml の algebraic effects を含む一部の構文を対象にしたステッパの実装を目指している。以下に Multicore OCaml の algebraic effects を含むプログラムのステップ表示の例を示す。

\begin{verbatim}
Input :  (try ((perform (E 1)) + (2 + (perform (E 4))))
          with | effect (E n) k -> (continue k n))
Step 1:  (continue (fun x => (try ((perform (E 1)) + (2 + x))
                              with | effect (E n) k -> (continue k n))) 4)
Step 2:  (try ((perform (E 1)) + (2 + 4))
          with | effect (E n) k -> (continue k n))
Step 3:  (try ((perform (E 1)) + 6)
          with | effect (E n) k -> (continue k n))
Step 4:  (continue (fun y => (try (y + 6)
                              with | effect (E n) k -> (continue k n))) 1)
Step 5:  (try (1 + 6) with | effect (E n) k -> (continue k n))
Step 6:  (try 7 with | effect (E n) k -> (continue k n))
Step 7:  7
\end{verbatim}

Multicore OCaml の「エフェクト」は \ref{section:definition} 節で定義した言語の algebraic effects のオペレーションとほとんど同じものだが、以下のような違いがある。

\begin{itemize}
\item (ハンドルの構文) \texttt{with \{return x -> er, op1(x; k) -> e1, ...\} handle e} は、Multicore OCaml では \texttt{match e with x -> er | effect (Op1 x) k -> e1 | ...\ } と書ける。
\item (宣言) \texttt{effect E : t1 -> t2} と書くことで、\texttt{t1} 型の引数をとって \texttt{t2} 型を返すエフェクト \texttt{E} が宣言できる。宣言していないエフェクトは使用できない。
\item (実行) エフェクトを呼び出す際には、 \texttt{perform (E e)} というように関数 \texttt{perform} に渡す。
\item (継続の適用) 継続 \texttt{k} に式 \texttt{e} を渡して実行を再開するには \texttt{continue k v} と書く。同じ継続は1度しか適用できない。
\end{itemize}

このような構文の違いはあるものの、継続が one-shot であることを除いて簡約のされかたは \ref{section:definition} 節で定めた言語のインタプリタと同様なので、インタプリタ関数を用意できれば変換によってステッパが導出できると考えられる。


\section{関連研究}

\cite{10.1145/2500365.2500590}
\cite{10.1007/978-3-030-02768-1_22}
\cite{e6cb0c3222794e48bf38cf44e46fe4aa}
\cite{10.1145/3158096}
\cite{PRETNAR201519}


\section{まとめ}
\label{section:conclusion}

ステッパを実装するためには、コンテキストの情報を保持しながら部分式を再帰的に実行するインタプリタを作ればよい。以前の研究\cite{FCA19}では言語ごとにコンテキストを表すデータ型を考えた上でインタプリタに実行の流れに従った新しい引数を付け足す作業が必要だったが、本研究では通常のインタプリタをCPS変換および非関数化するという機械的な操作でコンテキストの型およびコンテキストの情報を保持するインタプリタ関数を導出した。

その方法で、継続を明示的に扱える algebraic effects を含む言語に対するステッパを実装し、他の例外処理機能である try-with や shift/reset を含む言語についても同様の変換ができることを確認した。


\bibliographystyle{jplain}
\bibliography{stepper}

\newpage
\appendix
\section{付録}

\subsection{プログラムを再構成する関数}

\begin{figure}[h]
\begin{spacing}{0.9}
\begin{verbatim}
(* 式と handle 節内のコンテキスト情報を受け取って handle 節の式を再構成する *)
let rec plug_in_handle (e : e) (c : c) : e = match c with
  | FId -> e
  | FApp2 (e1, c) -> plug_in_handle (App (e1, e)) c
  | FApp1 (c, v2) -> plug_in_handle (App (e, Val v2)) c
  | FOp (name, c) -> plug_in_handle (Op (name, e)) c

(* 式と handle 節内と外のコンテキスト情報のペアを受け取ってプログラム全体を再構成する *)
let rec plug_all (e : e) ((c, c2) : (c * c2)) : e =
  let e_in_handle = plug_in_handle e c in
  match c2 with
  | GId -> e_in_handle
  | GHandle (h, c, c2) ->
    let e_handle = With (h, e_in_handle) in
    plug_all e_handle (c, c2)
\end{verbatim}
\end{spacing}
\caption{式とコンテキストを受け取って式全体を再構成する関数}
\label{figure:plug_all}
\end{figure}

\subsection{algebraic effect handlers のステッパによる実行の例}
\label{subsection:step_example}

\begin{figure}
\begin{spacing}{0.8}
\begin{verbatim}
Step 0:  ((with {return x -> (fun _ -> x),
                 Get(_; k) -> (fun s -> ((k s) s)),
                 Set(s; k) -> (fun _ -> ((k ()) s))}
           handle ((fun _ -> (Get ())) (Set ((Get ()) + 1)))) 0)
Step 1:  ((fun s ->
           (((fun y => (with {return x -> (fun _ -> x),
                              Get(_; k) -> (fun s -> ((k s) s)),
                              Set(s; k) -> (fun _ -> ((k ()) s))}
                        handle ((fun _ -> (Get ())) (Set (y + 1))))) s) s)) 0)
Step 2:  (((fun y => (with {return x -> (fun _ -> x),
                            Get(_; k) -> (fun s -> ((k s) s)),
                            Set(s; k) -> (fun _ -> ((k ()) s))}
                      handle ((fun _ -> (Get ())) (Set (y + 1))))) 0) 0)
Step 3:  ((with {return x -> (fun _ -> x),
                 Get(_; k) -> (fun s -> ((k s) s)),
                 Set(s; k) -> (fun _ -> ((k ()) s))}
           handle ((fun _ -> (Get ())) (Set (0 + 1)))) 0)
Step 4:  ((with {return x -> (fun _ -> x),
                 Get(_; k) -> (fun s -> ((k s) s)),
                 Set(s; k) -> (fun _ -> ((k ()) s))}
           handle ((fun _ -> (Get ())) (Set 1))) 0)
Step 5:  ((fun _ -> (((fun z => (with {return x -> (fun _ -> x),
                                       Get(_; k) -> (fun s -> ((k s) s)),
                                       Set(s; k) -> (fun _ -> ((k ()) s))}
                                 handle ((fun _ -> (Get ())) z))) ()) 1)) 0)
Step 6:  (((fun z => (with {return x -> (fun _ -> x),
                            Get(_; k) -> (fun s -> ((k s) s)),
                            Set(s; k) -> (fun _ -> ((k ()) s))}
                      handle ((fun _ -> (Get ())) z))) ()) 1)
Step 7:  ((with {return x -> (fun _ -> x),
                 Get(_; k) -> (fun s -> ((k s) s)),
                 Set(s; k) -> (fun _ -> ((k ()) s))}
           handle ((fun _ -> (Get ())) ())) 1)
Step 8:  ((with {return x -> (fun _ -> x),
                 Get(_; k) -> (fun s -> ((k s) s)),
                 Set(s; k) -> (fun _ -> ((k ()) s))} handle (Get ())) 1)
Step 9:  ((fun s -> (((fun a => (with {return x -> (fun _ -> x),
                                       Get(_; k) -> (fun s -> ((k s) s)),
                                       Set(s; k) -> (fun _ -> ((k ()) s))}
                                 handle a)) s) s)) 1)
Step 10:  (((fun a => (with {return x -> (fun _ -> x),
                             Get(_; k) -> (fun s -> ((k s) s)),
                             Set(s; k) -> (fun _ -> ((k ()) s))}
                       handle a)) 1) 1)
Step 11:  ((with {return x -> (fun _ -> x),
                  Get(_; k) -> (fun s -> ((k s) s)),
                  Set(s; k) -> (fun _ -> ((k ()) s))} handle 1) 1)
Step 12:  ((fun _ -> 1) 1)
Step 13:  1
\end{verbatim}
\end{spacing}
\caption{図 \ref{figure:6cps} のステッパを用いたステップ実行の例}
\label{figure:step_example}
\end{figure}

図\ref{figure:step_example} は、1つの値からなる状態を保持するハンドラを使ったプログラムをステップ実行した例である。

状態ハンドラは、受け取った値を記録して \texttt{()} を返す \texttt{Set} と、
\texttt{()} を受け取って記録された値を返す \texttt{Get} の 2 つのオペレーションを定義している。
\texttt{handle} 節で値が返された場合 (\texttt{return} 節) については、
状態にかかわらず値をそのまま返す。
入力プログラム (\texttt{Step 0} のプログラム) は、状態ハンドラのもとで、状態の初期値を \texttt{0} として、
\texttt{Set (Get () + 1)} をした後に \texttt{Get ()} を返すというプログラムである。

図 \ref{figure:step_example} の文字列は、
4 節で最終的に得られたステッパ (図\ref{figure:6cps}) を
整数、整数の足し算、unit 型で拡張したステッパの出力に改行とインデントを加えたものである。
プログラムが 1 段階ずつ書き換わっていて関数の内容も表示されているので、
限定継続の中身やその使われ方など理解が難しい機能の実行の様子を細かく見ることができる。
プログラムが長く読むのが大変な面はあるが、
\texttt{Step 3} でふたつ目の \texttt{Get ()} が \texttt{0} となり、
\texttt{Step 7} で \texttt{Set} により状態が \texttt{1} になり、
\texttt{Step 11} でひとつ目の \texttt{Get ()} が \texttt{1} になっているのが観察できる。

\newpage

\subsection{変換の各過程におけるインタプリタ}

\begin{figure}[!b]
\begin{spacing}{0.9}
  % code/3cps/eval.ml より
\begin{verbatim}
(* CPS インタプリタを非関数化して CPS 変換した関数 *)
let rec eval (exp : e) (k : k) (k2 : k2) : a = match exp with
  | Val (v) -> apply_in k v k2
  | App (e1, e2) -> eval e2 (FApp2 (e1, k)) k2
  | Op (name, e) -> eval e (FOp (name, k)) k2
  | With (h, e) ->
    eval e FId (fun a -> apply_handler k h a k2)  (* GHandle に変換される *)

(* handle 節内の継続を適用する関数 *)
and apply_in (k : k) (v : v) (k2 : k2) : a = match k with
  | FId -> k2 (Return v)  (* 継続を適用 *)
  | FApp2 (e1, k) -> let v2 = v in eval e1 (FApp1 (k, v2)) k2
  | FApp1 (k, v2) -> let v1 = v in
    (match v1 with
      | Fun (x, e) ->
        let reduct = subst e [(x, v2)] in
        eval reduct k k2
      | Cont (cont_value) ->
        (cont_value k) v2 k2
      | _ -> failwith "type error")
  | FOp (name, k) ->
    k2 (OpCall (name, v, (fun v -> fun k2' -> apply_in k v k2')))  (*継続を適用*)

(* handle 節内の実行結果をハンドラで処理する関数 *)
and apply_handler (k : k) (h : h) (a : a) (k2 : k2) : a = match a with
  | Return v ->
    (match h with {return = (x, e)} ->
      let reduct = subst e [(x, v)] in
      eval reduct k k2)
  | OpCall (name, v, m) ->
    (match search_op name h with
      | None ->
        k2 (OpCall (name, v, (fun v -> fun k2' ->  (* 継続を適用 *)
          m v (fun a' -> apply_handler k h a' k2'))))  (* GHandle に変換 *)
      | Some (x, y, e) ->
        let cont_value =
          Cont (fun k'' -> fun v -> fun k2'' ->
            m v (fun a' -> apply_handler k'' h a' k2'')) in  (* GHandle に変換 *)
        let reduct = subst e [(x, v); (y, cont_value)] in
        eval reduct k k2)

(* 初期継続を渡して実行を始める *)
let interpreter (e : e) : a = eval e FId (fun a -> a)  (* GId に変換される *)
\end{verbatim}
\caption{CPS インタプリタを非関数化して CPS 変換したプログラム}
\label{figure:3cps}
\end{spacing}
\end{figure}

\begin{figure}
\begin{spacing}{0.9}
  % code/4defun/syntax.ml より
\begin{verbatim}
(* CPS インタプリタを非関数化して CPS 変換して非関数化した関数 *)
let rec eval (exp : e) (k : k) (k2 : k2) : a = match exp with
  | Val (v) -> apply_in k v k2
  | App (e1, e2) -> eval e2 (FApp2 (e1, k)) k2
  | Op (name, e) -> eval e (FOp (name, k)) k2
  | With (h, e) -> eval e FId (GHandle (h, k, k2))

(* handle 節内の継続を適用する関数 *)
and apply_in (k : k) (v : v) (k2 : k2) : a = match k with
  | FId -> apply_out k2 (Return v)
  | FApp2 (e1, k) -> let v2 = v in eval e1 (FApp1 (k, v2)) k2
  | FApp1 (k, v2) -> let v1 = v in (match v1 with
    | Fun (x, e) ->
      let reduct = subst e [(x, v2)] in
      eval reduct k k2
    | Cont (cont_value) ->
      (cont_value k) v2 k2
    | _ -> failwith "type error")
  | FOp (name, k) ->
    apply_out k2 (OpCall (name, v, (fun v -> fun k2' -> apply_in k v k2')))

(* 全体の継続を適用する関数 *)
and apply_out (k2 : k2) (a : a) : a = match k2 with
  | GId -> a
  | GHandle (h, k, k2) -> apply_handler k h a k2

(* handle 節内の実行結果をハンドラで処理する関数 *)
and apply_handler (k : k) (h : h) (a : a) (k2 : k2) : a = match a with
  | Return v -> (match h with {return = (x, e)} ->
    let reduct = subst e [(x, v)] in eval reduct k k2)
  | OpCall (name, v, m) ->
    (match search_op name h with
      | None ->
        apply_out k2 (OpCall (name, v,
          (fun v -> fun k2' -> m v (GHandle (h, k, k2')))))
      | Some (x, y, e) ->
        let cont_value =
          Cont (fun k'' -> fun v -> fun k2 -> m v (GHandle (h, k'', k2))) in
        let reduct = subst e [(x, v); (y, cont_value)] in
        eval reduct k k2)

(* 初期継続を渡して実行を始める *)
let interpreter (e : e) : a = eval e FId GId
\end{verbatim}
\caption{CPS インタプリタを非関数化して CPS 変換して非関数化したプログラム}
\label{figure:4defun}
\end{spacing}
\end{figure}

\end{document}
