\section{言語とインタプリタの定義}
\label{section:definition}

この節では、型無し$\lambda$計算と algebraic effects からなる言語とそのインタプリタを定義する。

\subsection{algebraic effects}
\label{subsection:algebraic effects}

型無し$\lambda$計算と algebraic effects からなる対象言語を図\ref{figure:abstract_syntax}の \texttt{e} と定義する。ただし継続 \texttt{fun x => e} は入力プログラムに含まれることはなく、実行の過程で現れる。

\begin{figure}
\begin{verbatim}
v  :=                                      (値)
      x                                    変数
    | fun x -> e                           関数
    | fun x => e                           継続
e  :=                                      (式)
      v                                    値
    | e e                                  関数適用
    | op e                                 オペレーション呼び出し
    | with h handle e                      ハンドル
h  :=                                      (ハンドラ)
      {return x -> e,                      return 節
       op(x; k) -> e, ..., op(x; k) -> e}  オペレーション節(0個以上)
\end{verbatim}
\caption{対象言語の構文}
\label{figure:abstract_syntax}
\end{figure}

意味論については \ref{subsection:1cps} 節で厳密に扱うが、例えばプログラム \texttt{(with handler \{return x -> (fun a -> x), op(x, k) -> k x\} handle ((op (fun b -> b)) 2)) 3} は以下のように実行される。

right-to-left て評価するとして、まず関数適用の引数部分 \texttt{3} を実行し、値 \texttt{3} を得る。次に関数部分の \texttt{with ... handle ...} を実行する。\texttt{with ... handle ...} ではまず \texttt{handle} 以降の \texttt{((op (fun b -> b)) 2)} を実行する。これは関数適用なのでまず引数部分の \texttt{2} を実行し、値 \texttt{2} を得る。次に関数部分 \texttt{(op (fun b -> b))} はオペレーション呼び出しであり、まず引数 \texttt{(fun b -> b)} を実行する。値 \texttt{(fun b -> b)} を得たら、\texttt{(op (fun b -> b))} を囲んでいる \texttt{with ... handle ...} を参照して、そこにオペレーション \texttt{op} が定義されているかどうかを見る。ここでは \texttt{op(x, k) -> k x} と定義されている。すると、\texttt{with ... handle ...} の部分が \texttt{(k x)[(fun b -> b)/x, (fun y => (with handler \{return x -> (fun a -> x), op(x, k) -> k x\} handle (y 2)))/k} に簡約され、次はこの式の実行に移る。この \texttt{k} に代入する値は、 \texttt{with ... handle ...} 式までの限定継続である。引数部分と関数部分をそれぞれ実行し、これらは元々値になっているのでそのままの値が返る。そして \texttt{(fun b -> b)} に継続 \texttt{(fun y => (with ... handle (y 2)))} を適用すると \texttt{(with ... handle ((fun b -> b) 2))} になるが、この適用は継続の適用なので、\texttt{fun b -> b} および \texttt{2} は既に実行した後だということが分かっており、次には \texttt{(fun b -> b) 2} の関数適用をする。これは \texttt{2} になり、ここで \texttt{handle ...} 内が値となった。すると今度は \texttt{with ... handle ...} がハンドラの return 節の \texttt{fun a -> x} に簡約される。このとき \texttt{x} は \texttt{handle ...} 内の値の \texttt{2} である。よってここでプログラム全体は \texttt{(fun b -> 2) 3} になっており、この関数適用をして最終結果は \texttt{2} になる。

もし呼び出されたオペレーションがハンドラで定義されていなければさらに外のハンドラを参照して、その時の継続値は定義が見つかったハンドラまでの限定継続である。見つからなかった場合はエラーを起こす。

\subsection{構文の定義}
\label{subsection:syntax}

対象言語の OCaml による定義を図 \ref{figure:syntax} に示す。
式の型 \texttt{e} で表されるものが対象言語のプログラムである。

\begin{figure}
\begin{verbatim}
(* 値 *)
type v = Var of string      (* x *)
       | Fun of string * e  (* fun x -> e *)
       | Cont of (k -> k)   (* 継続 fun x => ... *)
(* ハンドラ *)
and h = {
  return : string * e;                       (* handler {return x -> e,      *)
  ops : (string * string * string * e) list  (*          op(x; k) -> e, ...} *)
}
(* 式 *)
and e = Val of v          (* v *)
      | App of e * e      (* e e *)
      | Op of string * e  (* op e *)
      | With of h * e     (* with h handle e *)
(* handle 内の継続 *)
and k = v -> a
(* handle 内の実行結果 *)
and a = Return of v               (* 値になった *)
      | OpCall of string * v * k  (* オペレーションが呼び出された *)
\end{verbatim}
\caption{対象言語の定義}
\label{figure:syntax}
\end{figure}

継続の型 \texttt{k} は、インタプリタ関数自体の継続を表す。

\subsection{CPS インタプリタ}
\label{subsection:1cps}

図 \ref{figure:syntax} の言語に対する、call-by-value かつ right-to-left のインタプリタを図 \ref{figure:1cps} に定義する。ただし、関数 \texttt{subst :\ e -> (string * v) list -> e} は代入のための関数であり、\texttt{subst e [(x, v); (k, cont\_value)]} は \texttt{e} の中の変数 \texttt{x} と変数 \texttt{k} に同時にそれぞれ値 \texttt{v} と値 \texttt{cont\_value} を代入した式を返す。関数 \texttt{search\_op} はハンドラ内のオペレーションを検索する関数で、例えば \texttt{handler \{return x -> x, op1(y, k) -> k y\}} を表すデータを \texttt{h} とすると \texttt{search\_op "op2" h} は \texttt{None} を返し \texttt{search\_op "op1" h} は \texttt{Some (y, k, App (Var "k", Var "y"))} を返す。

\begin{figure}
\begin{verbatim}
(* CPS インタプリタ *)
let rec eval (exp : e) (k : k) : a = match exp with
  | Val (v) -> k v  (* 継続に値を渡す *)
  | App (e1, e2) ->
    eval e2 (fun v2 ->  (* FApp2 に変換される関数 *)
        eval e1 (fun v1 -> match v1 with  (* FApp1 に変換される関数 *)
            | Fun (x, e) ->
              let reduct = subst e [(x, v2)] in  (* e[v2/x] *)
              eval reduct k
            | Cont (k') ->
              (k' k) v2  (* 現在の継続と継続値が保持するメタ継続を合成して値を渡す *)
            | _ -> failwith "type error"))
  | Op (name, e) ->
    eval e (fun v -> OpCall (name, v, k))  (* FOp に変換される関数 *)
  | With (h, e) ->
    let a = eval e (fun v -> Return v) in  (* FId に変換される関数、空の継続 *)
    apply_handler k h a  (* handle 節内の実行結果をハンドラで処理 *)

(* handle 節内の実行結果をハンドラで処理する関数 *)
and apply_handler (k : k) (h : h) (a : a) : a = match a with
  | Return v ->                         (* handle 節内が値 v を返したとき *)
    (match h with {return = (x, e)} ->  (* handler {return x -> e, ...} として*)
       let reduct = subst e [(x, v)] in (* e[v/x] に簡約される *)
       eval reduct k)                   (* e[v/x] を実行 *)
  | OpCall (name, v, k') ->        (* オペレーション呼び出しがあったとき *)
    (match search_op name h with
     | None ->                     (* ハンドラで定義されていない場合、 *)
       OpCall (name, v, (fun v ->  (* OpCall の継続の後に現在の継続を合成 *)
           let a' = k' v in
           apply_handler k h a'))  (* FCall に変換される関数 *)
     | Some (x, y, e) ->           (* ハンドラで定義されている場合、 *)
       let cont_value =
         Cont (fun k'' -> fun v -> (* 適用時にその後の継続を受け取って合成 *)
             let a' = k' v in
             apply_handler k'' h a') in  (* FCall に変換される関数 *)
       let reduct = subst e [(x, v); (y, cont_value)] in
       eval reduct k)

(* 初期継続を渡して実行を始める *)
let interpreter (e : e) : a = eval e (fun v -> Return v) (* FIdに変換される関数 *)
\end{verbatim}
\caption{継続渡し形式で書かれたインタプリタ}
\label{figure:1cps}
\end{figure}

このインタプリタは、\texttt{handle} 節内の実行については CPS になっており、メタ継続である \texttt{k} は「直近のハンドラまでの継続」である。関数 \texttt{eval} の下から2行目で再帰呼び出しの際に継続に \texttt{(fun x -> Return x)} を渡していて、これによって \texttt{handle} 節の実行に入るたびに渡す継続を初期化している。

そして \texttt{handle} 節内を実行した結果を表すのが \texttt{a} 型で、 \texttt{handle} 節を実行した結果値 \texttt{v} になったことを \texttt{Return v} と表し、値になる前にオペレーション呼び出しが行われたことを \texttt{OpCall (name, v, k)} で表す。この結果とハンドラを受け取ってハンドラの処理をするのが関数 \texttt{apply\_handler} である。関数 \texttt{apply\_handler} の動作は \texttt{handle} 節の実行結果とハンドラの内容によって3種類ある。

\begin{enumerate}
\item \texttt{handle} 節が値 \texttt{v}になった場合:ハンドラの return 節 \texttt{return x -> e} を参照して、\texttt{e[v/x]} を実行
\item \texttt{handle} 節がオペレーション呼び出し \texttt{OpCall (name, v, k')} になった場合で、そのオペレーション \texttt{name} がハンドラ内で定義されていなかった場合:\texttt{handle} 節内の限定継続 \texttt{k'} に、1つ外側の \texttt{handle} までの限定継続を合成した継続 \texttt{fun v -> ...} を作り、それを \texttt{OpCall (name, v, (fun v -> ...))} と返す
\item \texttt{handle} 節がオペレーション呼び出し \texttt{OpCall (name, v, k')} になった場合で、そのオペレーション \texttt{name} がハンドラ内で定義されていた場合:そのハンドラの定義 \texttt{name (x; y) -> e} を参照し、 \texttt{e[v/x, k'/y]} を実行
\end{enumerate}

関数 \texttt{apply\_handler} の下から 5 行目に \texttt{(fun k'' -> fun v -> ...)} という関数がある。このうち \texttt{fun v -> ...} の部分は \texttt{k'} の後に \texttt{k''} を行うという継続であり、継続を実際に適用する際に引数に適用するものである。\texttt{k''} はこの継続を適用する時点で関数 \texttt{eval} が引数に保持している継続である。なぜ \texttt{k''} をとる関数になっているのかというと、例えば \texttt{(fun a -> a) ((fun y => with h handle (fun b -> b) y) 1)} の継続の適用をするとき、継続値は「受け取った値に \texttt{fun b -> b} を適用してそれをハンドラ \texttt{h} で処理する」というメタ継続を持っているが、さらにその後には「そこに\texttt{fun a -> a}を適用する」という継続を合成したいのに、継続を capture する時点では \texttt{fun a -> a} についての情報は見えないからである。関数 \texttt{eval} 内の \texttt{(k' k) v2} という部分で、この関数 \texttt{(fun k'' -> fun v -> ...)} に継続の適用の後の継続 \texttt{k} を渡すことで合成された継続を作り、そこに値を渡している。
