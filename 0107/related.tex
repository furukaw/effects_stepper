\section{関連研究}

PLT Redex \cite{felleisen09}は操作的意味論の形式化のための言語で、文法と簡約規則を定義できるようになっており、DrRacket のステッパを継承している。さらに、一画面の中に各ステップでのプログラムを配置し、あるプログラムが1ステップ簡約されて別のプログラムになることを矢印で表したグラフを表示する。これはより視覚的に簡約の様子を表すことができるほか、各矢印にそのステップの簡約規則が添えられているのでよりステップを辿りやすい。しかし、複数のステップのプログラムが同じ画面に表示されている上に簡約が起こる部分式が強調されていないので、長いプログラムをステップ実行すると見づらくなってしまう。

根岸ら\cite{NI2009}の関数型言語 Haskell のデバッガフロントエンドは、一般的なデバッガの実行方法が通用しない遅延評価型言語のグラフィカルなユーザインタフェースでのステップ実行を含むデバッガ操作を可能にした。デバッガがブラウザ上で利用できるようになっており、DrRacket や本研究のステッパと同様に各ステップでのプログラムを、評価中の式をハイライトしながら表示する。実行するファイルや、ブレイクポイントをどの関数に設定するか、ステップ実行するか次のブレイクポイントまで実行するか、などといった設定をブラウザ上のボタンなどをクリックすることで行うことができる。デバッガのインタフェースでの本研究との違いは、根岸ら\cite{NI2009}のデバッガではブレイクポイントをユーザが簡単に設定できるのに対して、本研究ではブレイクポイントは自動的に全ての簡約基に設定され、ユーザは詳細に実行のしかたを決められない代わりに「次のステップ」「前のステップ」などのボタンを押すのみのより簡単な操作のみでステップ実行をすることができる。

Tunnel Wilson et al. \cite{tunnell18} は、代数学的ステッパの出力のような内容を学生に手書きで書かせることによって、学生がプログラムの実行のされかたをどのように理解しているか、どのような構文のステップの書き下しができないかといった傾向を分析した。
