\section{はじめに}

書いたプログラムが思った通りの挙動をしない時、プログラマはデバッグをする必要がある。

単純なデバッグはプログラムを実行した際の出力から推測したりソースコードを眺めることで行われるが、そのようなデバッグは「ソースコードのどの部分が間違っているか」を示すものが無く、多くの時間や労力を要することがある。特にプログラミングにまだ慣れていない初学者にとっては、デバッグの経験や言語に対する理解が乏しい為、より困難な作業になると考えられる。

そこで色々な言語にデバッガが用意されているが、デバッガを利用するには、デバッガのコマンドの文字列や意味を覚えたり、ブレイクポイントを設定する箇所を考えたりといった、初学者にとってやはり困難な操作が必要になる。また、一般的なデバッガで表示されるのは「ソースコード中の実行中の行」であり、どこで今の関数を呼び出されたのか、この後どんな計算があるのかなどといったプログラム全体の流れが分かりにくい。

\begin{figure}
  \begin{center}
    \includegraphics[width=13cm]{racket1.png}
  \end{center}
  \caption{DrRacket のステッパ}
  \label{figure:racket1}
\end{figure}

\begin{figure}
  \begin{center}
    \includegraphics[width=13cm]{racket2.png}
  \end{center}
  \caption{DrRacket のステッパを進めた様子}
  \label{figure:racket2}
\end{figure}

我々は、プログラミング初心者がデバッグをするのに最適な方法は、ステッパを使うことだと考える。ステッパは Racket 言語の統合開発環境 DrRacket において提供されているツール\cite{clements01}である。ユーザがエディタにプログラムを書いてステッパ起動ボタンを押すと、図\ref{figure:racket1}のようなウインドウが表示される。図\ref{figure:racket1}は、再帰関数を用いて2の階乗を計算するプログラムを入力してステッパを起動したときの様子である。ウインドウには左右にそれぞれプログラムが表示されている。左はユーザが入力したプログラムと同じものであり、このプログラムで最初に簡約される式 \texttt{(fact 2)} が緑色にハイライトされている。右側のプログラムでは、ハイライトされた部分以外は左側と同じプログラムが表示されており、左側では緑色だった式 \texttt{(fact 2)} がその簡約結果に置き換えられ、紫色でハイライトされている。

Step ボタンのうち右の実行を進めるボタンを押すと図\ref{figure:racket2}のような表示に切り替わる。最初(図\ref{figure:racket1})は右側にあったプログラムと同じプログラムが左に表示され、次に簡約される部分式 \texttt{(= 2 0)} が緑色にハイライトされており、右側には同様にその部分が簡約されて紫色になったプログラムが表示されている。当初 \texttt{(fact 2)} だった式がその値である \texttt{2} になるステップまで、ボタンを押すと次々に簡約が行われてプログラムが変形していく様子を視覚的に見ることができる。

このように、プログラムを実行したときに、実行結果の値だけでなく、実行中にプログラムが代数的にどのように書き換えられていくかを見せるツールがステッパである。ステッパの操作は基本的に「前のステップへ」「次のステップへ」のボタンを押すのみであり、プログラミングや CUI での操作に慣れていない初心者でも使いやすい。

しかし、DrRacket のステッパが受け付けるのは Racket 言語のうちの一部の構文で構成された教育用の言語であり、例外処理などがサポートされていない。初心者にとって理解しにくい例外処理をステップ実行できるようにするため、著者らは関数型言語 OCaml の、try-with を含む基礎的な構文に対応したステッパを実装し評価した\cite{FCA19}。

本研究の目的は、継続を明示的に扱うことができる言語機能を含む言語に対応したステッパを作ることである。
shift/reset \ref{?} や algebraic effects \ref{?} といった継続を扱う言語機能を含むプログラムの挙動は複雑で理解が困難だが、継続を値として扱うことができる言語機能に対応したステッパはまだ作られていない。

ステッパは簡約により書き換わっていくプログラム全体を出力するインタプリタなので、ステッパを実装する際には、部分式を再帰的に実行している時にそのコンテキストの情報を得ることが必要になる。以前の研究\cite{FCA19}では評価順序をもとにコンテキストの構造を考えてコンテキストを表すデータ型を定義していたが、本研究ではインタプリタ関数に CPS 変換 \cite{?} および非関数化 \cite{?} という変換を施すことでコンテキストの情報を引数に持つインタプリタ関数を導出し、それを用いてプログラム全体を再構成して出力する方法を提案する。
