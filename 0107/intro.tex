\section{はじめに}

継続を操作するプログラムの挙動を理解するためには、プログラムがどのように実行されていくのかを詳細に追う必要がある。
その継続を得た時点での実行の状態が分からなければどのような継続なのかが分からないからである。

そのようなプログラムの実行の様子を確認する方法に、実行が1段階進むごとにどのような状態になっているのかを書き連ねる方法がある。1段階ずつ確認していけば、具体的にどのような計算がされるのかを観察することができ、プログラムの動きを理解しやすい。またこの方法はデバッグにおいても有用で、どの段階で想定と違うことが起こっているのかが見えるので、プログラムのどの部分がその原因なのかが分かりやすくなる。

そこで我々は、プログラムが代数的に書き換わる様子を1ステップずつ表示するツールである代数的ステッパを、継続を明示的に扱う言語機能に対応させることを目指している。
ステッパは、例えば OCaml のプログラム \texttt{let a = 1 + 2 in 4 + a} を入力されると、以下のような実行ステップを表す文字列を出力する。

\begin{quote}
\begin{verbatim}
Step 0:  (let a = (1 + 2) in (4 + a))
Step 1:  (let a = 3 in (4 + a))

Step 1:  (let a = 3 in (4 + a))
Step 2:  (4 + 3)

Step 2:  (4 + 3)
Step 3:  7
\end{verbatim}
\end{quote}

ステッパはプログラムを評価するプログラムなのでインタプリタの一種である。古川ら(2019)\cite{FCA19}は、例外処理のための構文 try-with を含む言語について、通常のインタプリタ関数に出力機能を追加することでステッパを実装した。インタプリタは部分式を再帰的に実行するが、その先で簡約があった場合に式全体を出力するために、部分式を再帰的に実行している時にそのコンテキストの情報を得ることが必要になる。古川ら(2019)\cite{FCA19}は評価順序をもとにコンテキストの構造を考えてコンテキストを表すデータ型を定義していたが、本研究ではインタプリタ関数に CPS 変換 \cite{PLOTKIN1975125} および非関数化 \cite{Reynolds1998} という変換を施すことでコンテキストの情報を引数に持つインタプリタ関数を導出し、それを用いてプログラム全体を再構成して出力する方法を提案する。そして実際に algebraic effects を含む言語についてインタプリタ関数を定義し、それを変換することによってステッパ関数を導出する過程について説明する。
