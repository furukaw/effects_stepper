\section{はじめに}

プログラムの実行の様子を確認する方法に、実行が1段階進むごとにどのよう
な状態になっているのかを書き連ねる方法がある。このツールは、代数的ス
テッパ(以後、単にステッパと書く)と呼ばれ、プログラムが代数的に書き換
わる様子を1ステップずつ表示してくれるものである。
例えば OCaml のプログラム \texttt{let a = 1 + 2 in 4 + a} を入力される
と、ステッパは以下のような実行ステップを表す文字列を出力する。

\begin{verbatim}
Step 0:  (let a = (1 + 2) in (4 + a))
Step 1:  (let a = 3 in (4 + a))

Step 1:  (let a = 3 in (4 + a))
Step 2:  (4 + 3)

Step 2:  (4 + 3)
Step 3:  7
\end{verbatim}

このように1段階ずつ確認していけば、具体的にどのような計算がされるのか
を観察することができ、プログラムの動きを理解しやすい。またこの方法はデ
バッグにおいても有用で、どの段階で想定と違うことが起こっているのかが見
えるので、プログラムのどの部分がその原因なのかが分かりやすくなる。その
ため、お茶の水女子大学では実際に関数型言語の授業でステッパを本格的に使
用している。

しかし、継続を操作するようなプログラムだった場合、ステッパをどのように
作ったら良いのかは明らかではない。そのような複雑なプログラムでこそ実行
の様子を詳細に追いたいところだが、現在のところそのような言語に対するス
テッパは作られていない。
わずかに、我々が過去の研究 \cite{FCA19} で例外処理のための
構文 try-with を含む言語についてのステッパを実装した程度である。

そこで我々は、ステッパを、継続を明示的に扱う言語機能に対応させることを
目指している。ステッパは、通常のインタプリタ関数に出力機能を追加するこ
とで実装できるが、その際、式全体を再構成するためにコンテキストの情報が
必要になる。以前の研究 \cite{FCA19} では、部分式の簡約に進むたびにコン
テキストを表すデータ型を作成していたが、継続を操作するようなプログラム
の場合、どのようなコンテキストにすれば良いのかは即座には明らかではな
い。

そこで、本論文では、インタプリタに対して CPS 変換 \cite{PLOTKIN1975125} と
非関数化 \cite{Reynolds1998} を施すことで機械的にコンテキストの情報を
得る。この方法を使うと、継続を操作するような言語でも機械的にコンテキス
トの情報を得ることができ、それを使ってステッパを作ることができるように
なる。本論文では、この手法を algebraic effects を含む言語に対して適用
し、algebraic effects を含む言語に対するステッパを作成する。また、その
過程で algebraic effects を含む言語に対する definitional interpreter
を示す。本論文では詳しくは述べないが、この手法は shift/reset に対する
ステッパの作成にも使うことができる。

本論文の構成は以下の通りである。
まず \ref{section:context} 節でステッパを実装する方法を紹介する。
そして \ref{section:definition} 節で algebraic effects を含む言語およびそのインタプリタを定義し、
\ref{section:transform} 節でインタプリタを変換してステッパを得るまでの過程を説明する。
\ref{section:languages} 節では他のいくつかの言語に対するステッパを同様の変換によって得ることについて議論する。
\ref{section:related} 節で関連研究について触れ、\ref{section:conclusion} 節でまとめる。

%継続を操作するプログラムの挙動を理解するためには、プログラムがどのように実行されていくのかを詳細に追う必要がある。
%その継続を得た時点での実行の状態が分からなければどのような継続なのかが分からないからである。

%そのようなプログラムの実行の様子を確認する方法に、実行が1段階進むごとにどのような状態になっているのかを書き連ねる方法がある。1段階ずつ確認していけば、具体的にどのような計算がされるのかを観察することができ、プログラムの動きを理解しやすい。またこの方法はデバッグにおいても有用で、どの段階で想定と違うことが起こっているのかが見えるので、プログラムのどの部分がその原因なのかが分かりやすくなる。

%そこで我々は、プログラムが代数的に書き換わる様子を1ステップずつ表示するツールである代数的ステッパを、継続を明示的に扱う言語機能に対応させることを目指している。
%ステッパは、例えば OCaml のプログラム \texttt{let a = 1 + 2 in 4 + a} を入力されると、以下のような実行ステップを表す文字列を出力する。

%\begin{quote}
%\begin{verbatim}
%Step 0:  (let a = (1 + 2) in (4 + a))
%Step 1:  (let a = 3 in (4 + a))
%
%Step 1:  (let a = 3 in (4 + a))
%Step 2:  (4 + 3)
%
%Step 2:  (4 + 3)
%Step 3:  7
%\end{verbatim}
%\end{quote}

%ステッパはプログラムを評価するプログラムなのでインタプリタの一種である。古川ら(2019)\cite{FCA19}は、例外処理のための構文 try-with を含む言語について、通常のインタプリタ関数に出力機能を追加することでステッパを実装した。インタプリタは部分式を再帰的に実行するが、その先で簡約があった場合に式全体を出力するために、部分式を再帰的に実行している時にそのコンテキストの情報を得ることが必要になる。古川ら(2019)\cite{FCA19}は評価順序をもとにコンテキストの構造を考えてコンテキストを表すデータ型を定義していたが、本研究ではインタプリタ関数に CPS 変換 \cite{PLOTKIN1975125} および非関数化 \cite{Reynolds1998} という変換を施すことでコンテキストの情報を引数に持つインタプリタ関数を導出し、それを用いてプログラム全体を再構成して出力する方法を提案する。そして実際に algebraic effects を含む言語についてインタプリタ関数を定義し、それを変換することによってステッパ関数を導出する過程について説明する。
